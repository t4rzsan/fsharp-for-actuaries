
% Default to the notebook output style

    


% Inherit from the specified cell style.




    
\documentclass[11pt]{article}

    
    
    \usepackage[T1]{fontenc}
    % Nicer default font (+ math font) than Computer Modern for most use cases
    \usepackage{mathpazo}

    % Basic figure setup, for now with no caption control since it's done
    % automatically by Pandoc (which extracts ![](path) syntax from Markdown).
    \usepackage{graphicx}
    % We will generate all images so they have a width \maxwidth. This means
    % that they will get their normal width if they fit onto the page, but
    % are scaled down if they would overflow the margins.
    \makeatletter
    \def\maxwidth{\ifdim\Gin@nat@width>\linewidth\linewidth
    \else\Gin@nat@width\fi}
    \makeatother
    \let\Oldincludegraphics\includegraphics
    % Set max figure width to be 80% of text width, for now hardcoded.
    \renewcommand{\includegraphics}[1]{\Oldincludegraphics[width=.8\maxwidth]{#1}}
    % Ensure that by default, figures have no caption (until we provide a
    % proper Figure object with a Caption API and a way to capture that
    % in the conversion process - todo).
    \usepackage{caption}
    \DeclareCaptionLabelFormat{nolabel}{}
    \captionsetup{labelformat=nolabel}

    \usepackage{adjustbox} % Used to constrain images to a maximum size 
    \usepackage{xcolor} % Allow colors to be defined
    \usepackage{enumerate} % Needed for markdown enumerations to work
    \usepackage{geometry} % Used to adjust the document margins
    \usepackage{amsmath} % Equations
    \usepackage{amssymb} % Equations
    \usepackage{textcomp} % defines textquotesingle
    % Hack from http://tex.stackexchange.com/a/47451/13684:
    \AtBeginDocument{%
        \def\PYZsq{\textquotesingle}% Upright quotes in Pygmentized code
    }
    \usepackage{upquote} % Upright quotes for verbatim code
    \usepackage{eurosym} % defines \euro
    \usepackage[mathletters]{ucs} % Extended unicode (utf-8) support
    \usepackage[utf8x]{inputenc} % Allow utf-8 characters in the tex document
    \usepackage{fancyvrb} % verbatim replacement that allows latex
    \usepackage{grffile} % extends the file name processing of package graphics 
                         % to support a larger range 
    % The hyperref package gives us a pdf with properly built
    % internal navigation ('pdf bookmarks' for the table of contents,
    % internal cross-reference links, web links for URLs, etc.)
    \usepackage{hyperref}
    \usepackage{longtable} % longtable support required by pandoc >1.10
    \usepackage{booktabs}  % table support for pandoc > 1.12.2
    \usepackage[inline]{enumitem} % IRkernel/repr support (it uses the enumerate* environment)
    \usepackage[normalem]{ulem} % ulem is needed to support strikethroughs (\sout)
                                % normalem makes italics be italics, not underlines
    

    
    
    % Colors for the hyperref package
    \definecolor{urlcolor}{rgb}{0,.145,.698}
    \definecolor{linkcolor}{rgb}{.71,0.21,0.01}
    \definecolor{citecolor}{rgb}{.12,.54,.11}

    % ANSI colors
    \definecolor{ansi-black}{HTML}{3E424D}
    \definecolor{ansi-black-intense}{HTML}{282C36}
    \definecolor{ansi-red}{HTML}{E75C58}
    \definecolor{ansi-red-intense}{HTML}{B22B31}
    \definecolor{ansi-green}{HTML}{00A250}
    \definecolor{ansi-green-intense}{HTML}{007427}
    \definecolor{ansi-yellow}{HTML}{DDB62B}
    \definecolor{ansi-yellow-intense}{HTML}{B27D12}
    \definecolor{ansi-blue}{HTML}{208FFB}
    \definecolor{ansi-blue-intense}{HTML}{0065CA}
    \definecolor{ansi-magenta}{HTML}{D160C4}
    \definecolor{ansi-magenta-intense}{HTML}{A03196}
    \definecolor{ansi-cyan}{HTML}{60C6C8}
    \definecolor{ansi-cyan-intense}{HTML}{258F8F}
    \definecolor{ansi-white}{HTML}{C5C1B4}
    \definecolor{ansi-white-intense}{HTML}{A1A6B2}

    % commands and environments needed by pandoc snippets
    % extracted from the output of `pandoc -s`
    \providecommand{\tightlist}{%
      \setlength{\itemsep}{0pt}\setlength{\parskip}{0pt}}
    \DefineVerbatimEnvironment{Highlighting}{Verbatim}{commandchars=\\\{\}}
    % Add ',fontsize=\small' for more characters per line
    \newenvironment{Shaded}{}{}
    \newcommand{\KeywordTok}[1]{\textcolor[rgb]{0.00,0.44,0.13}{\textbf{{#1}}}}
    \newcommand{\DataTypeTok}[1]{\textcolor[rgb]{0.56,0.13,0.00}{{#1}}}
    \newcommand{\DecValTok}[1]{\textcolor[rgb]{0.25,0.63,0.44}{{#1}}}
    \newcommand{\BaseNTok}[1]{\textcolor[rgb]{0.25,0.63,0.44}{{#1}}}
    \newcommand{\FloatTok}[1]{\textcolor[rgb]{0.25,0.63,0.44}{{#1}}}
    \newcommand{\CharTok}[1]{\textcolor[rgb]{0.25,0.44,0.63}{{#1}}}
    \newcommand{\StringTok}[1]{\textcolor[rgb]{0.25,0.44,0.63}{{#1}}}
    \newcommand{\CommentTok}[1]{\textcolor[rgb]{0.38,0.63,0.69}{\textit{{#1}}}}
    \newcommand{\OtherTok}[1]{\textcolor[rgb]{0.00,0.44,0.13}{{#1}}}
    \newcommand{\AlertTok}[1]{\textcolor[rgb]{1.00,0.00,0.00}{\textbf{{#1}}}}
    \newcommand{\FunctionTok}[1]{\textcolor[rgb]{0.02,0.16,0.49}{{#1}}}
    \newcommand{\RegionMarkerTok}[1]{{#1}}
    \newcommand{\ErrorTok}[1]{\textcolor[rgb]{1.00,0.00,0.00}{\textbf{{#1}}}}
    \newcommand{\NormalTok}[1]{{#1}}
    
    % Additional commands for more recent versions of Pandoc
    \newcommand{\ConstantTok}[1]{\textcolor[rgb]{0.53,0.00,0.00}{{#1}}}
    \newcommand{\SpecialCharTok}[1]{\textcolor[rgb]{0.25,0.44,0.63}{{#1}}}
    \newcommand{\VerbatimStringTok}[1]{\textcolor[rgb]{0.25,0.44,0.63}{{#1}}}
    \newcommand{\SpecialStringTok}[1]{\textcolor[rgb]{0.73,0.40,0.53}{{#1}}}
    \newcommand{\ImportTok}[1]{{#1}}
    \newcommand{\DocumentationTok}[1]{\textcolor[rgb]{0.73,0.13,0.13}{\textit{{#1}}}}
    \newcommand{\AnnotationTok}[1]{\textcolor[rgb]{0.38,0.63,0.69}{\textbf{\textit{{#1}}}}}
    \newcommand{\CommentVarTok}[1]{\textcolor[rgb]{0.38,0.63,0.69}{\textbf{\textit{{#1}}}}}
    \newcommand{\VariableTok}[1]{\textcolor[rgb]{0.10,0.09,0.49}{{#1}}}
    \newcommand{\ControlFlowTok}[1]{\textcolor[rgb]{0.00,0.44,0.13}{\textbf{{#1}}}}
    \newcommand{\OperatorTok}[1]{\textcolor[rgb]{0.40,0.40,0.40}{{#1}}}
    \newcommand{\BuiltInTok}[1]{{#1}}
    \newcommand{\ExtensionTok}[1]{{#1}}
    \newcommand{\PreprocessorTok}[1]{\textcolor[rgb]{0.74,0.48,0.00}{{#1}}}
    \newcommand{\AttributeTok}[1]{\textcolor[rgb]{0.49,0.56,0.16}{{#1}}}
    \newcommand{\InformationTok}[1]{\textcolor[rgb]{0.38,0.63,0.69}{\textbf{\textit{{#1}}}}}
    \newcommand{\WarningTok}[1]{\textcolor[rgb]{0.38,0.63,0.69}{\textbf{\textit{{#1}}}}}
    
    
    % Define a nice break command that doesn't care if a line doesn't already
    % exist.
    \def\br{\hspace*{\fill} \\* }
    % Math Jax compatability definitions
    \def\gt{>}
    \def\lt{<}
    % Document parameters
    \title{2. Structuring Data}
    
    
    

    % Pygments definitions
    
\makeatletter
\def\PY@reset{\let\PY@it=\relax \let\PY@bf=\relax%
    \let\PY@ul=\relax \let\PY@tc=\relax%
    \let\PY@bc=\relax \let\PY@ff=\relax}
\def\PY@tok#1{\csname PY@tok@#1\endcsname}
\def\PY@toks#1+{\ifx\relax#1\empty\else%
    \PY@tok{#1}\expandafter\PY@toks\fi}
\def\PY@do#1{\PY@bc{\PY@tc{\PY@ul{%
    \PY@it{\PY@bf{\PY@ff{#1}}}}}}}
\def\PY#1#2{\PY@reset\PY@toks#1+\relax+\PY@do{#2}}

\expandafter\def\csname PY@tok@w\endcsname{\def\PY@tc##1{\textcolor[rgb]{0.73,0.73,0.73}{##1}}}
\expandafter\def\csname PY@tok@c\endcsname{\let\PY@it=\textit\def\PY@tc##1{\textcolor[rgb]{0.25,0.50,0.50}{##1}}}
\expandafter\def\csname PY@tok@cp\endcsname{\def\PY@tc##1{\textcolor[rgb]{0.74,0.48,0.00}{##1}}}
\expandafter\def\csname PY@tok@k\endcsname{\let\PY@bf=\textbf\def\PY@tc##1{\textcolor[rgb]{0.00,0.50,0.00}{##1}}}
\expandafter\def\csname PY@tok@kp\endcsname{\def\PY@tc##1{\textcolor[rgb]{0.00,0.50,0.00}{##1}}}
\expandafter\def\csname PY@tok@kt\endcsname{\def\PY@tc##1{\textcolor[rgb]{0.69,0.00,0.25}{##1}}}
\expandafter\def\csname PY@tok@o\endcsname{\def\PY@tc##1{\textcolor[rgb]{0.40,0.40,0.40}{##1}}}
\expandafter\def\csname PY@tok@ow\endcsname{\let\PY@bf=\textbf\def\PY@tc##1{\textcolor[rgb]{0.67,0.13,1.00}{##1}}}
\expandafter\def\csname PY@tok@nb\endcsname{\def\PY@tc##1{\textcolor[rgb]{0.00,0.50,0.00}{##1}}}
\expandafter\def\csname PY@tok@nf\endcsname{\def\PY@tc##1{\textcolor[rgb]{0.00,0.00,1.00}{##1}}}
\expandafter\def\csname PY@tok@nc\endcsname{\let\PY@bf=\textbf\def\PY@tc##1{\textcolor[rgb]{0.00,0.00,1.00}{##1}}}
\expandafter\def\csname PY@tok@nn\endcsname{\let\PY@bf=\textbf\def\PY@tc##1{\textcolor[rgb]{0.00,0.00,1.00}{##1}}}
\expandafter\def\csname PY@tok@ne\endcsname{\let\PY@bf=\textbf\def\PY@tc##1{\textcolor[rgb]{0.82,0.25,0.23}{##1}}}
\expandafter\def\csname PY@tok@nv\endcsname{\def\PY@tc##1{\textcolor[rgb]{0.10,0.09,0.49}{##1}}}
\expandafter\def\csname PY@tok@no\endcsname{\def\PY@tc##1{\textcolor[rgb]{0.53,0.00,0.00}{##1}}}
\expandafter\def\csname PY@tok@nl\endcsname{\def\PY@tc##1{\textcolor[rgb]{0.63,0.63,0.00}{##1}}}
\expandafter\def\csname PY@tok@ni\endcsname{\let\PY@bf=\textbf\def\PY@tc##1{\textcolor[rgb]{0.60,0.60,0.60}{##1}}}
\expandafter\def\csname PY@tok@na\endcsname{\def\PY@tc##1{\textcolor[rgb]{0.49,0.56,0.16}{##1}}}
\expandafter\def\csname PY@tok@nt\endcsname{\let\PY@bf=\textbf\def\PY@tc##1{\textcolor[rgb]{0.00,0.50,0.00}{##1}}}
\expandafter\def\csname PY@tok@nd\endcsname{\def\PY@tc##1{\textcolor[rgb]{0.67,0.13,1.00}{##1}}}
\expandafter\def\csname PY@tok@s\endcsname{\def\PY@tc##1{\textcolor[rgb]{0.73,0.13,0.13}{##1}}}
\expandafter\def\csname PY@tok@sd\endcsname{\let\PY@it=\textit\def\PY@tc##1{\textcolor[rgb]{0.73,0.13,0.13}{##1}}}
\expandafter\def\csname PY@tok@si\endcsname{\let\PY@bf=\textbf\def\PY@tc##1{\textcolor[rgb]{0.73,0.40,0.53}{##1}}}
\expandafter\def\csname PY@tok@se\endcsname{\let\PY@bf=\textbf\def\PY@tc##1{\textcolor[rgb]{0.73,0.40,0.13}{##1}}}
\expandafter\def\csname PY@tok@sr\endcsname{\def\PY@tc##1{\textcolor[rgb]{0.73,0.40,0.53}{##1}}}
\expandafter\def\csname PY@tok@ss\endcsname{\def\PY@tc##1{\textcolor[rgb]{0.10,0.09,0.49}{##1}}}
\expandafter\def\csname PY@tok@sx\endcsname{\def\PY@tc##1{\textcolor[rgb]{0.00,0.50,0.00}{##1}}}
\expandafter\def\csname PY@tok@m\endcsname{\def\PY@tc##1{\textcolor[rgb]{0.40,0.40,0.40}{##1}}}
\expandafter\def\csname PY@tok@gh\endcsname{\let\PY@bf=\textbf\def\PY@tc##1{\textcolor[rgb]{0.00,0.00,0.50}{##1}}}
\expandafter\def\csname PY@tok@gu\endcsname{\let\PY@bf=\textbf\def\PY@tc##1{\textcolor[rgb]{0.50,0.00,0.50}{##1}}}
\expandafter\def\csname PY@tok@gd\endcsname{\def\PY@tc##1{\textcolor[rgb]{0.63,0.00,0.00}{##1}}}
\expandafter\def\csname PY@tok@gi\endcsname{\def\PY@tc##1{\textcolor[rgb]{0.00,0.63,0.00}{##1}}}
\expandafter\def\csname PY@tok@gr\endcsname{\def\PY@tc##1{\textcolor[rgb]{1.00,0.00,0.00}{##1}}}
\expandafter\def\csname PY@tok@ge\endcsname{\let\PY@it=\textit}
\expandafter\def\csname PY@tok@gs\endcsname{\let\PY@bf=\textbf}
\expandafter\def\csname PY@tok@gp\endcsname{\let\PY@bf=\textbf\def\PY@tc##1{\textcolor[rgb]{0.00,0.00,0.50}{##1}}}
\expandafter\def\csname PY@tok@go\endcsname{\def\PY@tc##1{\textcolor[rgb]{0.53,0.53,0.53}{##1}}}
\expandafter\def\csname PY@tok@gt\endcsname{\def\PY@tc##1{\textcolor[rgb]{0.00,0.27,0.87}{##1}}}
\expandafter\def\csname PY@tok@err\endcsname{\def\PY@bc##1{\setlength{\fboxsep}{0pt}\fcolorbox[rgb]{1.00,0.00,0.00}{1,1,1}{\strut ##1}}}
\expandafter\def\csname PY@tok@kc\endcsname{\let\PY@bf=\textbf\def\PY@tc##1{\textcolor[rgb]{0.00,0.50,0.00}{##1}}}
\expandafter\def\csname PY@tok@kd\endcsname{\let\PY@bf=\textbf\def\PY@tc##1{\textcolor[rgb]{0.00,0.50,0.00}{##1}}}
\expandafter\def\csname PY@tok@kn\endcsname{\let\PY@bf=\textbf\def\PY@tc##1{\textcolor[rgb]{0.00,0.50,0.00}{##1}}}
\expandafter\def\csname PY@tok@kr\endcsname{\let\PY@bf=\textbf\def\PY@tc##1{\textcolor[rgb]{0.00,0.50,0.00}{##1}}}
\expandafter\def\csname PY@tok@bp\endcsname{\def\PY@tc##1{\textcolor[rgb]{0.00,0.50,0.00}{##1}}}
\expandafter\def\csname PY@tok@fm\endcsname{\def\PY@tc##1{\textcolor[rgb]{0.00,0.00,1.00}{##1}}}
\expandafter\def\csname PY@tok@vc\endcsname{\def\PY@tc##1{\textcolor[rgb]{0.10,0.09,0.49}{##1}}}
\expandafter\def\csname PY@tok@vg\endcsname{\def\PY@tc##1{\textcolor[rgb]{0.10,0.09,0.49}{##1}}}
\expandafter\def\csname PY@tok@vi\endcsname{\def\PY@tc##1{\textcolor[rgb]{0.10,0.09,0.49}{##1}}}
\expandafter\def\csname PY@tok@vm\endcsname{\def\PY@tc##1{\textcolor[rgb]{0.10,0.09,0.49}{##1}}}
\expandafter\def\csname PY@tok@sa\endcsname{\def\PY@tc##1{\textcolor[rgb]{0.73,0.13,0.13}{##1}}}
\expandafter\def\csname PY@tok@sb\endcsname{\def\PY@tc##1{\textcolor[rgb]{0.73,0.13,0.13}{##1}}}
\expandafter\def\csname PY@tok@sc\endcsname{\def\PY@tc##1{\textcolor[rgb]{0.73,0.13,0.13}{##1}}}
\expandafter\def\csname PY@tok@dl\endcsname{\def\PY@tc##1{\textcolor[rgb]{0.73,0.13,0.13}{##1}}}
\expandafter\def\csname PY@tok@s2\endcsname{\def\PY@tc##1{\textcolor[rgb]{0.73,0.13,0.13}{##1}}}
\expandafter\def\csname PY@tok@sh\endcsname{\def\PY@tc##1{\textcolor[rgb]{0.73,0.13,0.13}{##1}}}
\expandafter\def\csname PY@tok@s1\endcsname{\def\PY@tc##1{\textcolor[rgb]{0.73,0.13,0.13}{##1}}}
\expandafter\def\csname PY@tok@mb\endcsname{\def\PY@tc##1{\textcolor[rgb]{0.40,0.40,0.40}{##1}}}
\expandafter\def\csname PY@tok@mf\endcsname{\def\PY@tc##1{\textcolor[rgb]{0.40,0.40,0.40}{##1}}}
\expandafter\def\csname PY@tok@mh\endcsname{\def\PY@tc##1{\textcolor[rgb]{0.40,0.40,0.40}{##1}}}
\expandafter\def\csname PY@tok@mi\endcsname{\def\PY@tc##1{\textcolor[rgb]{0.40,0.40,0.40}{##1}}}
\expandafter\def\csname PY@tok@il\endcsname{\def\PY@tc##1{\textcolor[rgb]{0.40,0.40,0.40}{##1}}}
\expandafter\def\csname PY@tok@mo\endcsname{\def\PY@tc##1{\textcolor[rgb]{0.40,0.40,0.40}{##1}}}
\expandafter\def\csname PY@tok@ch\endcsname{\let\PY@it=\textit\def\PY@tc##1{\textcolor[rgb]{0.25,0.50,0.50}{##1}}}
\expandafter\def\csname PY@tok@cm\endcsname{\let\PY@it=\textit\def\PY@tc##1{\textcolor[rgb]{0.25,0.50,0.50}{##1}}}
\expandafter\def\csname PY@tok@cpf\endcsname{\let\PY@it=\textit\def\PY@tc##1{\textcolor[rgb]{0.25,0.50,0.50}{##1}}}
\expandafter\def\csname PY@tok@c1\endcsname{\let\PY@it=\textit\def\PY@tc##1{\textcolor[rgb]{0.25,0.50,0.50}{##1}}}
\expandafter\def\csname PY@tok@cs\endcsname{\let\PY@it=\textit\def\PY@tc##1{\textcolor[rgb]{0.25,0.50,0.50}{##1}}}

\def\PYZbs{\char`\\}
\def\PYZus{\char`\_}
\def\PYZob{\char`\{}
\def\PYZcb{\char`\}}
\def\PYZca{\char`\^}
\def\PYZam{\char`\&}
\def\PYZlt{\char`\<}
\def\PYZgt{\char`\>}
\def\PYZsh{\char`\#}
\def\PYZpc{\char`\%}
\def\PYZdl{\char`\$}
\def\PYZhy{\char`\-}
\def\PYZsq{\char`\'}
\def\PYZdq{\char`\"}
\def\PYZti{\char`\~}
% for compatibility with earlier versions
\def\PYZat{@}
\def\PYZlb{[}
\def\PYZrb{]}
\makeatother


    % Exact colors from NB
    \definecolor{incolor}{rgb}{0.0, 0.0, 0.5}
    \definecolor{outcolor}{rgb}{0.545, 0.0, 0.0}



    
    % Prevent overflowing lines due to hard-to-break entities
    \sloppy 
    % Setup hyperref package
    \hypersetup{
      breaklinks=true,  % so long urls are correctly broken across lines
      colorlinks=true,
      urlcolor=urlcolor,
      linkcolor=linkcolor,
      citecolor=citecolor,
      }
    % Slightly bigger margins than the latex defaults
    
    \geometry{verbose,tmargin=1in,bmargin=1in,lmargin=1in,rmargin=1in}
    
    

    \begin{document}
    
    
    \maketitle
    
    

    
    \section{Structuring Data}\label{structuring-data}

    In Excel you typically organize data in tables where each column is of a
a specific \emph{type} like number, text string or date. In R you use
vector, matrix, list and array and your values are of \emph{type}
Logical, Numeric, Integer or Character. Similarly, F\# has \emph{types}.

F\# has a lot of built-in typs such as \texttt{string}, \texttt{int} and
\texttt{decimal} and it also has container types like \texttt{array},
\texttt{list} and \texttt{seq} (sequence) that can contain other items.
The .NET framework itself also has a lot of types that you can use, such
as the commonly used \texttt{DateTime} type. The best part is that you
can create your own types.

\subsection{Records}\label{records}

You have already seen some of the built-in F\# types and you have also
created some types of your own in the first example, where you created
the \texttt{PersonPolicy} type.

    \begin{Verbatim}[commandchars=\\\{\}]
{\color{incolor}In [{\color{incolor}35}]:} \PY{k}{type} \PY{n+nc}{PersonPolicy} \PY{o}{=} 
             \PY{o}{\PYZob{}}
                 \PY{n}{PersonId}\PY{o}{:} \PY{k+kt}{string}\PY{o}{;}
                 \PY{n}{PolicyNumber}\PY{o}{:} \PY{k+kt}{string}\PY{o}{;}
                 \PY{n}{Premium}\PY{o}{:} \PY{k+kt}{decimal}\PY{o}{;}
             \PY{o}{\PYZcb{}}
\end{Verbatim}


    As you can see, \texttt{PersonPolicy} is really just a combination of
other types: \texttt{string} and \texttt{decimal}. That way you can
create an endless number of types by combining existing types. This kind
of type is called a \emph{record} type. It is also called a
\emph{product} type because its sample space is
\texttt{string\ *\ string\ *\ decimal}.

You create values of a type by using the \texttt{let} keyword.

    \begin{Verbatim}[commandchars=\\\{\}]
{\color{incolor}In [{\color{incolor}36}]:} \PY{k}{let} \PY{n+nv}{theAnswer} \PY{o}{=} \PY{l+m+mi}{42} \PY{c+c1}{// An integer}
         \PY{k}{let} \PY{n+nv}{greeting} \PY{o}{=} \PY{l+s}{\PYZdq{}}\PY{l+s}{Hello world}\PY{l+s}{\PYZdq{}} \PY{c+c1}{// A string}
         \PY{k}{let} \PY{n+nv}{pi} \PY{o}{=} \PY{l+m+mi}{3}\PY{o}{.}\PY{l+m+mi}{141} \PY{c+c1}{// A double}
         \PY{k}{let} \PY{n+nv}{pp} \PY{o}{=} 
             \PY{o}{\PYZob{}}
                 \PY{n}{PersonId} \PY{o}{=} \PY{l+s}{\PYZdq{}}\PY{l+s}{123}\PY{l+s}{\PYZdq{}}\PY{o}{;}
                 \PY{n}{PolicyNumber} \PY{o}{=} \PY{l+s}{\PYZdq{}}\PY{l+s}{Pol001}\PY{l+s}{\PYZdq{}}\PY{o}{;}
                 \PY{n}{Premium} \PY{o}{=} \PY{l+m+mi}{10000m}\PY{o}{;}
             \PY{o}{\PYZcb{}}
\end{Verbatim}


    Notice how you don't have to specify the type anywhere. You just create
the value and most of the time F\# will figure out what type you
intended. F\# will keep track of the types behind the scenes which is
very useful when defining functions as you will see later.

\begin{quote}
Why the \texttt{m}? Notice the \texttt{m} after the premium amount in
line 8 above? The \texttt{m} tells F\# that you want a decimal and not
an integer. If you remove the \texttt{m}, you will get an error saying
that F\# cannot convert the integer 10000 to a decimal, since the
Premium field is of type decimal.
\end{quote}

\subsection{Discriminated Unions}\label{discriminated-unions}

Discriminated unions are a way of defining a type with mutually
exclusive ways of creating values of that type. It sounds weird but it
is a really nice way to represent data. Say for example that you have to
policy systems in your company. One is an old legacy system where policy
numbers are represented as integers. For the other newer system policy
numbers are strings. For this setup you might define the PersonPolicy
like so.

    \begin{Verbatim}[commandchars=\\\{\}]
{\color{incolor}In [{\color{incolor}37}]:} \PY{k}{type} \PY{n+nc}{PolicyNumber} \PY{o}{=}
             \PY{o}{|} \PY{n}{LegacyPolicyNumber} \PY{k}{of} \PY{n}{int}
             \PY{o}{|} \PY{n}{NewPolicyNumber} \PY{k}{of} \PY{k+kt}{string}
             
         \PY{k}{type} \PY{n+nc}{PersonPolicy2} \PY{o}{=}
             \PY{o}{\PYZob{}}
                 \PY{n}{PersonId}\PY{o}{:} \PY{k+kt}{string}\PY{o}{;}
                 \PY{n}{PolicyNumber}\PY{o}{:} \PY{n}{PolicyNumber}\PY{o}{;}
                 \PY{n}{Premium}\PY{o}{:} \PY{k+kt}{decimal}\PY{o}{;}
             \PY{o}{\PYZcb{}}
\end{Verbatim}


    The \texttt{PersonPolicy2} type is used like so.

    \begin{Verbatim}[commandchars=\\\{\}]
{\color{incolor}In [{\color{incolor}38}]:} \PY{k}{let} \PY{n+nv}{legacyPolicyPerson} \PY{o}{=} 
             \PY{o}{\PYZob{}}
                 \PY{n}{PersonId} \PY{o}{=} \PY{l+s}{\PYZdq{}}\PY{l+s}{1}\PY{l+s}{\PYZdq{}}\PY{o}{;}
                 \PY{n}{PolicyNumber} \PY{o}{=} \PY{n}{LegacyPolicyNumber}\PY{o}{(}\PY{l+m+mi}{42}\PY{o}{)}\PY{o}{;}
                 \PY{n}{Premium} \PY{o}{=} \PY{l+m+mi}{1000m}\PY{o}{;}
             \PY{o}{\PYZcb{}}
             
         \PY{k}{let} \PY{n+nv}{newPolicyPerson} \PY{o}{=}
             \PY{o}{\PYZob{}}
                 \PY{n}{PersonId} \PY{o}{=} \PY{l+s}{\PYZdq{}}\PY{l+s}{2}\PY{l+s}{\PYZdq{}}\PY{o}{;}
                 \PY{n}{PolicyNumber} \PY{o}{=} \PY{n}{NewPolicyNumber}\PY{o}{(}\PY{l+s}{\PYZdq{}}\PY{l+s}{Pol01}\PY{l+s}{\PYZdq{}}\PY{o}{)}\PY{o}{;}
                 \PY{n}{Premium} \PY{o}{=} \PY{l+m+mi}{1200m}\PY{o}{;}    
             \PY{o}{\PYZcb{}}
         
         \PY{n}{display}\PY{o}{(}\PY{n}{legacyPolicyPerson}\PY{o}{.}\PY{n}{PolicyNumber}\PY{o}{)}
         \PY{n}{display}\PY{o}{(}\PY{n}{newPolicyPerson}\PY{o}{.}\PY{n}{PolicyNumber}\PY{o}{)}
\end{Verbatim}


    
    
    
    
    Using discriminated unions like that gives you complete control and type
safety when handling data. In the above example, the discriminated union
ensures that there is no doubt whether you are holding a legacy policy
number or a new policy number. The type tells us what it is. Later on
you will learn how to use \emph{matching} to handle discriminated
unions.

\subsubsection{Options}\label{options}

Probably the most important discriminated union in F\# is the
\href{https://docs.microsoft.com/en-us/dotnet/fsharp/language-reference/options}{\texttt{Option}}
type. An option can be either \texttt{Some} or \texttt{None} where
\texttt{None} means that the value does not exist. For example, you
would use \texttt{None} if a value read from a CSV file or from a
database is missing. Option values are set like so.

    \begin{Verbatim}[commandchars=\\\{\}]
{\color{incolor}In [{\color{incolor}39}]:} \PY{k}{let} \PY{n+nv}{existingValue} \PY{o}{=} \PY{n}{Some}\PY{o}{(}\PY{l+m+mi}{42}\PY{o}{)}
         \PY{k}{let} \PY{n+nv}{missingValue} \PY{o}{=} \PY{n}{None}
\end{Verbatim}


    \subsubsection{Single Case Discriminated
Unions}\label{single-case-discriminated-unions}

    Let us say you have a function that creates a displayname from given
name and surname (we will get to functions in the next chapter).

    \begin{Verbatim}[commandchars=\\\{\}]
{\color{incolor}In [{\color{incolor}40}]:} \PY{k}{let} \PY{n+nv}{createDisplayName} \PY{n}{givenName} \PY{n}{surName} \PY{o}{=}
             \PY{n}{givenName} \PY{o}{+} \PY{l+s}{\PYZdq{}}\PY{l+s}{ }\PY{l+s}{\PYZdq{}} \PY{o}{+} \PY{n}{surName}
             
         \PY{k}{let} \PY{n+nv}{a} \PY{o}{=} \PY{l+s}{\PYZdq{}}\PY{l+s}{Jakob}\PY{l+s}{\PYZdq{}}
         \PY{k}{let} \PY{n+nv}{b} \PY{o}{=} \PY{l+s}{\PYZdq{}}\PY{l+s}{Christensen}\PY{l+s}{\PYZdq{}}
         \PY{n}{createDisplayName} \PY{n}{a} \PY{n}{b}
\end{Verbatim}


\begin{Verbatim}[commandchars=\\\{\}]
{\color{outcolor}Out[{\color{outcolor}40}]:} Jakob Christensen
\end{Verbatim}
            
    Accidentally, you may call it like this because both parameters are of
type \texttt{string} and therefore interchangeable.

    \begin{Verbatim}[commandchars=\\\{\}]
{\color{incolor}In [{\color{incolor}41}]:} \PY{n}{createDisplayName} \PY{n}{b} \PY{n}{a}
\end{Verbatim}


\begin{Verbatim}[commandchars=\\\{\}]
{\color{outcolor}Out[{\color{outcolor}41}]:} Christensen Jakob
\end{Verbatim}
            
    To make it harder for the caller to make this mistake, you can introduce
single case discriminated unions.

    \begin{Verbatim}[commandchars=\\\{\}]
{\color{incolor}In [{\color{incolor}44}]:} \PY{c+c1}{// GivenName and SurName are single case discriminated unions}
         \PY{k}{type} \PY{n+nc}{GivenName} \PY{o}{=} \PY{n}{GivenName} \PY{k}{of} \PY{k+kt}{string}
         \PY{k}{type} \PY{n+nc}{SurName} \PY{o}{=} \PY{n}{SurName} \PY{k}{of} \PY{k+kt}{string}
         
         \PY{c+c1}{// \PYZdq{}Deconstruct\PYZdq{} givenName and surName to get the actual string values inside.}
         \PY{k}{let} \PY{n+nv}{createDisplayName2} \PY{o}{(}\PY{n}{GivenName} \PY{n}{givenName}\PY{o}{)} \PY{o}{(}\PY{n}{SurName} \PY{n}{surName}\PY{o}{)} \PY{o}{=}
             \PY{n}{givenName} \PY{o}{+} \PY{l+s}{\PYZdq{}}\PY{l+s}{ }\PY{l+s}{\PYZdq{}} \PY{o}{+} \PY{n}{surName}
             
         \PY{c+c1}{// \PYZdq{}Construct\PYZdq{} a GivenName and a SurName}
         \PY{k}{let} \PY{n+nv}{a2} \PY{o}{=} \PY{o}{(}\PY{n}{GivenName} \PY{l+s}{\PYZdq{}}\PY{l+s}{Jakob}\PY{l+s}{\PYZdq{}}\PY{o}{)}
         \PY{k}{let} \PY{n+nv}{b2} \PY{o}{=} \PY{o}{(}\PY{n}{SurName} \PY{l+s}{\PYZdq{}}\PY{l+s}{Christensen}\PY{l+s}{\PYZdq{}}\PY{o}{)}
         \PY{n}{createDisplayName2} \PY{n}{a2} \PY{n}{b2}
\end{Verbatim}


\begin{Verbatim}[commandchars=\\\{\}]
{\color{outcolor}Out[{\color{outcolor}44}]:} Jakob Christensen
\end{Verbatim}
            
    If you accidentally switch the two arguments, you will get an error
because the types \texttt{GivenName} and \texttt{SurName} are not
considered the same by F\#, even though they are actually just strings.

    \begin{Verbatim}[commandchars=\\\{\}]
{\color{incolor}In [{\color{incolor}43}]:} \PY{n}{createDisplayName2} \PY{n}{b2} \PY{n}{a2}
\end{Verbatim}


    \begin{Verbatim}[commandchars=\\\{\}]
Stopped due to error

    \end{Verbatim}

    \begin{Verbatim}[commandchars=\\\{\}]

        input.fsx (1,20)-(1,22) typecheck error This expression was expected to have type
        'GivenName'    
    but here has type
        'SurName'    
    input.fsx (1,23)-(1,25) typecheck error This expression was expected to have type
        'SurName'    
    but here has type
        'GivenName'    

    \end{Verbatim}

    If you want to get the value "inside" a single case discriminated union,
you need to deconstruct it. The function \texttt{createDisplayName2}
above shows how to do that easily for function parameters. If you want
to desconstruct without doing it as a function parameter, it is a bit
more cumbersome. This is how it is done.

    \begin{Verbatim}[commandchars=\\\{\}]
{\color{incolor}In [{\color{incolor}46}]:} \PY{k}{let} \PY{o}{(}\PY{n}{GivenName} \PY{n}{deconstructedGivenName}\PY{o}{)} \PY{o}{=} \PY{n}{a2}
         \PY{n}{deconstructedGivenName}
\end{Verbatim}


\begin{Verbatim}[commandchars=\\\{\}]
{\color{outcolor}Out[{\color{outcolor}46}]:} Jakob
\end{Verbatim}
            

    % Add a bibliography block to the postdoc
    
    
    
    \end{document}
